\documentclass[12pt,margin=0.5in]{article}

\usepackage{titlesec}
\usepackage[toc,page]{appendix}
\usepackage{graphicx}
\usepackage[margin=1in]{geometry}
\usepackage{placeins}
\usepackage{caption}
\usepackage{listings}
\usepackage{hyperref} % active links
\usepackage[usenames]{color} % add colors
\usepackage{float}
\usepackage{subfiles} % subfiles 

\definecolor{mygray}{gray}{0.8}

\lstset{breaklines=true,
		breakatwhitespace=true,
		numbersep=10pt,
		basicstyle=\footnotesize,
		columns=fullflexible,
		backgroundcolor=\color{mygray},
		frame=single}

\setcounter{secnumdepth}{5}
\setcounter{tocdepth}{5}
\newcommand{\sectionbreak}{\clearpage}
\newcommand{\newparagraph}[1]{\paragraph{#1}\mbox{}\\}
\newcommand{\newoption}[1]{\mbox{}\\{\it #1}}

\graphicspath{{./figs/}{./data/}}

\title{GeoPAT 2.0 user's manual\newline {\normalsize {\bf Notice:} At present this is a very incomplete manual but it has installation instructions.}} 
\author{\url{http://sil.uc.edu}}
\date{\today}

\begin{document}
\maketitle
\newpage

\tableofcontents
\newpage

\section{Introduction}
\subfile{sections/01_introduction}

\section{GeoPAT 2.0 architecture}
\subfile{sections/02_architecture}

\section{Basic workflow paths with examples}
\subfile{sections/03_workflows}

\FloatBarrier

\begin{appendices}

\section{GeoPAT 2.0 installation}
\subfile{sections/04_installation}

\section{Numerical signatures and normalization methods available in GeoPAT} \label{signatures}
\subfile{sections/05_normalization}

\section{Dissimilarity measures available in GeoPAT}
\subfile{sections/06_measures}

\end{appendices}

\bibliographystyle{abbrv}
\bibliography{main}

\end{document}
