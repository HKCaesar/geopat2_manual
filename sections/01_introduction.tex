\documentclass[../GeoPAT2_Manual.tex]{subfiles}

GeoPAT (Geospatial Pattern Analysis Toolbox) is a standalone suite of modules written in C and dedicated to analysis of large Earth Science datasets in their entirety using spatial and/or temporal patterns. 
Global scale, high resolution spatial datasets are available but are mostly used in small pieces for local studies. 
GeoPAT enables studying them in their entirety.
GeoPAT’s core idea is to tessellate global spatial data into grid of square blocks of original cells (pixels).
This transforms data from its original form (huge number of cells each having simple content) to a new form (much smaller number of supercells/blocks with complex content).
Complex cell contains a pattern of original variable.
GeoPAT provides means for succinct description of such patterns and for calculation of similarity between patterns.
This enables spatial analysis such as clustering, segmentation, and search to be performed on the grid of complex cells (local patterns).

\subsection{Basic concepts}

\begin{itemize}
  \item \textbf{Grid} is a topological structure applied to input data.
Grid is a lattice of cells defined by its geographical position, spatial extent, number of cells, and cell size which defines grid resolution.
Each cell in the grid contains a vector of values.
  \item \textbf{Pattern signatures} - compact numerical descriptors of patterns. 
  We use histograms of pattern’s features as scene signatures.
  \item \textbf{Motifel} is an elementary unit of analysis - a square block of pixels representing local pattern.
Motifel is represented by histogram of features (co-occurrence, decomposition).
Distance between motifels is a distance between their histograms (for example using Jensen-Shannon Divergence). 
\end{itemize}
